\documentclass{article}
\usepackage[utf8]{inputenc}

\title{Trabalho de Física}
\author{Cristhian Grundmann\\Danilo Lemos Cardoso}
\date{October 2019}

\begin{document}

\maketitle

\section*{Introdução}

Foi feita a simulação de um elástico. O elástico é discretizado em uma sequência finita de pontos, de tal modo que dois pontos consecutivos se comportem como uma mola:
$$ F = -k^2x $$
onde $x$ é a variação em relação à uma distância normal,
$F$ é a intensidade da força e
$k$ é uma constante.

Cada ponto possui três atributos: posição, velocidade e aceleração.

A equação diferencial de segunda ordem foi estimada numericamente por dois métodos: o método e Euler e o método do ponto médio.

Há um formulário interativo que permite a manipulação de variáveis, como número de pontos, gravidade, constante elástica, e o método de aproximação a ser usado.

\section*{Método de Euler}
O método de Euler foi implementado da seguinte maneira.

A posição de um ponto é atualizada somando-a com sua velocidade(multiplicada por um intervalo de tempo fixo). A velocidade é atualizada somando-a com sua aceleração. A aceleração é calculada, com exatidão, através das novas posições dos pontos.

\section*{Método do ponto médio}
O método do ponto médio foi implementado da seguinte maneira.

A posição é atualizada segundo o método de Euler, as para a metade do intervalo do tempo. A aceleração dos pontos é calculada com exatidão. A velocidade dos pontos é atualizada com o método de Euler com metade do intervalo do tempo. A posição é atualizada a partir da posição inicial e da velocidade, usando Euler com intervalo de tempo inteiro.

\section*{Colisão com círculos}
Para melhor entretenimento, foi adicionado um esquema de colisão entre o elástico e círculos fixos, de modo que o elástico possa deslizar.

A colisão é testada ponto por ponto. Caso um ponto esteja dentro de um círculo, sua posição é alterada radialmente para ficar na circunferência e a componente da velocidade na direção radial invertida e multiplicada por uma constante específica do círculo.

\section*{Conclusão}

O experimento permitiu uma comparação entre os dois métodos, e ficou evidente que o método de Euler acumula erros de forma explosiva.




\end{document}
